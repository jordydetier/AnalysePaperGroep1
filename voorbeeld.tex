%==============================================================================
% Voorbeeld gebruik documentklasse hogent-article
%==============================================================================
%
% Compileren in TeXstudio:
%
% - Zorg dat Biber de bibliografie compileert (en niet Biblatex)
%   Options > Configure > Build > Default Bibliography Tool: "txs:///biber"
% - F5 om te compileren en het resultaat te bekijken.
% - Als de bibliografie niet zichtbaar is, probeer dan F5 - F8 - F5
%   Met F8 compileer je de bibliografie apart.
%
% Als je JabRef gebruikt voor het bijhouden van de bibliografie, zorg dan
% dat je in ``biblatex''-modus opslaat: File > Switch to BibLaTeX mode.

\documentclass{hogent-article}
\bibliography{bibliografie}
%\usepackage{lipsum} % Voor vultekst

%------------------------------------------------------------------------------
% Metadata over het artikel
%------------------------------------------------------------------------------

%---------- Titel & auteur ----------------------------------------------------

% TODO: geef werktitel van je eigen voorstel op
\PaperTitle{De invloed van een business analist op agile projecten}
% TODO: geef op welk soort artikel dit is
% Dit is typisch de opdracht en het vak waarvoor dit artikel geschreven is, bv.
% ``Verslag onderzoeksproject Onderzoekstechnieken 2018-2019''
\PaperType{Verslag onderzoeksproject Analyse 2019-2020, Klas: 3D, Groep : 1}

% TODO: vul je eigen naam in als auteur, geef ook je emailadres mee!
\Authors{  Bram Vanoverbeke\textsuperscript{1}, 
	       Stef Verlinde\textsuperscript{2},
           Jordy De Tier\textsuperscript{3},
           Joppe Minjauw\textsuperscript{4}
} % Authors

% TODO: vul de naam van je co-promotor in.
% Als het hier gaat om een voorstel voor de bachelorproef, dan ben je hier
% verplicht de naam van je co-promotor in te vullen. Zoniet, dan kan je het
% leeg laten.
\CoPromotor{}

% Contactinfo: Geef hier de contactgegevens van elke auteur van het artikel (en
% indien van toepassing ook van de co-promotor)
\affiliation{
	\textsuperscript{1} \href{mailto:bram.vanoverbeke@student.hogent.be}{bram.vanoverbeke@student.hogent.be}
	\textsuperscript{2}
	\href{mailto:stef.verlinde@student.hogent.be}{	stef.verlinde@student.hogent.be}
}
\affiliation{
	\textsuperscript{3}
	\href{mailto:jordy.detier@student.hogent.be}{jordy.detier@student.hogent.be}
	\textsuperscript{4} \href{mailto:joppe.minjauw@student.hogent.be}{joppe.minjauw@student.hogent.be}
}

%---------- Abstract ----------------------------------------------------------

\Abstract{In deze paper wordt onderzocht wat de invloed is van een business analist op het agile werken binnen een project. Daarnaast wordt nagegaan of business analist de dag van vandaag nog relevant is als aparte rol, of eerder als skill bij andere rollen kan toegevoegd worden. Om dit te onderzoeken hebben we, naast het lezen van al gepubliceerde papers, een aantal bedrijven gecontacteerd om na te gaan hoe het er daar aan toe gaat. We vermoeden dat de taken van de business analist meer als skill bij andere rollen binnen het project worden toegevoegd dan vroeger, maar dat afhankelijk van het project een business analist zeker een nodige aanwinst blijft. 
}

%---------- Onderzoeksdomein en sleutelwoorden --------------------------------
% TODO: Vul de sleutelwoorden aan.


\Keywords{Agile werken; Business analist;}
\newcommand{\keywordname}{Sleutelwoorden} % Defines the keywords heading name

%---------- Titel, inhoud -----------------------------------------------------

\begin{document}

\flushbottom % Makes all text pages the same height
\maketitle % Print the title and abstract box
\tableofcontents % Print the contents section
\thispagestyle{empty} % Removes page numbering from the first page

%------------------------------------------------------------------------------
% Hoofdtekst
%------------------------------------------------------------------------------

\section{Inleiding}


Agile werken is een alom gekend begrip binnen de bedrijfswereld en is tegenwoordig bij de meeste bedrijfsprocessen niet meer weg te denken. De nood aan digitalisering zit daar zeker voor een stuk tussen, maar wat houdt agile werken precies in en, specifieker, wat is de rol van een business analist?
Aan de hand van eerder gepubliceerde papers en interviews met business analisten die ervaring hebben met het agile proces, wordt onderzocht waar de business analist voor in staat en of deze rol nog steeds relevant is binnen hedendaagse bedrijven.
\newline
\newline Agile werken zorgt ervoor dat, zoals de letterlijke vertaling 'behendig, flexibel' al doet vermoeden, een team gemakkelijk kan inspelen op veranderingen van omstandigheden binnen een project, bijvoordeeld nieuwe eisen van de klant, zonder dat het eindresultaat daardoor in gevaar komt. In tegenstelling tot de watervalmethode worden bij agile werken grote productontwikkelingen opgedeeld in verschillende sprints van twee tot vier weken. Binnen elke sprint wordt een voorafgekozen onderdeel van het eindproduct onder handen genomen, wat wil zeggen dat het alle processen doorloopt, van analyse tot testing. Door op deze manier te werk te gaan kan na elke sprint een werkend deel van het eindproduct voorgelegd worden aan de klant, wat feedback en aanpassing gemakkelijker maakt dan wanneer het volledige product op het einde pas aan de klant getoond wordt. Ook stelt agile werken het team in staat mogelijke fouten al vroeg uit het product te halen en zo desastreuze problemen te vermijden.\newline
\newline Wat zijn nu de taken van de business analist binnen dit agile proces? Wel, voor een business analist draait het grotendeels rond de user stories. Een user story is een korte, eenvoudige beschrijving van een behoefte van de eindgebruiker, geschreven vanuit het standpunt van deze eindgebruiker. Zie het als een kort verhaaltje dat verstaanbaar is voor zowel de klant als het projectteam. Het is de taak van de business analist om, na meermalig overleg met de klant, deze user stories op te stellen en op te volgen, zodat de juiste requirements worden doorgegeven aan het projectteam. Daarnaast staat de business analist ook in voor het beheren van de product backlog. In het algemeen kan de business analist gezien worden als de schakel tussen de klant en het projectteam.

\section{Literatuurstudie}

% Refereren naar de literatuur kan met:
% \autocite{BIBTEXKEY} -> (Auteur, jaartal)
% \textcite{BIBTEXKEY} -> Auteur (jaartal)

'Agile Bi - The Future of Bi'\autocite{Muntean2013} is een onderzoek die beschreef wat agile werken eigenlijk is, waarom het nuttig is voor Business Intelligence en beschreef kort op welke manieren het kan toegepast worden. In dit onderzoek bleek dat uit alle agile methodes Scrum toch wel een van de populairste was en zeker voor het oplossen van de problemen bij BI projecten. Zo wordt een project onderverdeeld in user stories die dan binnen een sprint van 1-2 weken ontworpen, ontwikkeld en getest worden. Het voordeel hierbij is dat de gebruikers veel meer betrokken worden binnen het project aangezien het projectteam na elke sprint een nieuwe deliverable kan tonen. Naast het belang van agile werken in het development gedeelte, toont dit onderzoek ook aan dat op vlak van business analyse agile werken zeer interessant is. Zo zal volgens het onderzoek het maken van spreadsheets, persentaties en andere gebruikte middelen in de business analyse veel duidelijker en makkelijker te volgen zijn voor anderen. Al bij al blijkt uit dit onderzoek dat agile werken zeer lucratief is. Zo kunnen bedrijven bijvoorbeeld veel makkelijker en beter inspringen op hun snel veranderende markt. \newline\newline Wanneer we het onderzoek van \autocite{Muntean2013} vergelijken met andere onderzoeken rond agile projecten zoals het onderzoek: 'Distributed agile: project management in a global environment' \autocite{LeeYong2009}. Dan zien we dat agile werken toch meer en meer opkomt. Vooral het aangename resultaat in productiviteit en omzet komt dit ten goede.  Dit gaat wel ten koste van de business-analist. In bijna elk onderzoek wordt business-analyse bijna tot helemaal niet meer vermeld.\newline \newline Uit de literatuurstudie en enkele nuttige bronnen kon er veel geleerd worden over hoe agile projecten nu eigenlijk worden toegepast. Toch werd een business-analist bijna nooit vermeld in deze bronnen en is er dus twijfel of deze wel degelijk een nut heeft. Onze hypthose luidt dus als volgt:\newline

\underline{\textbf{Hypothese :}}\newline
$H_0$: De business-analist speelt geen belangrijke rol in agile projecten en is dus verwaarloosbaar. \\
$H_1$: De business-analist speelt een belangrijke rol in agile projecten en is dus noodzakelijk \\ \bigskip
\section{Vragen voor het interview}
Voor het opstellen van dit onderzoek zochten we ook naar enkele bedrijven en personen om ons wat meer informatie te geven. Zo stelden we enkele vragen die ons wat meer duidelijkheid zouden moeten brengen.\newline \newline Als eerste vraag stelden we : \textit{"Wat is de taak van een business-analist?"}. Deze vraag vonden we belangrijk om ons een concreet beeld te geven van de functie als business-analist, zodat er geen verwarring onstaat.\newline\newline De tweede vraag luidde: \textit{"Is business-analyse in agile projecten nog altijd een skill of een rol?"}. Met deze vraag probeerden we te weten te komen of business-analyse in agile projecten wel degelijk top priority is en dus een rol is of eerder een nice to have skill is.\newline \newline De derde vraag : \textit{Hoe verloopt het agile proces binnen jullie bedrijf?} werd gesteld om te kijken hoe agile werken nu eigenlijk toegepast wordt in het bedrijf.\newline De volgende vraag : \textit{Wat vindt u voor en nadelen van agile werken?} was belangrijk om zo ook de mindere kanten van agile werken te weten te komen.\newline \newline Als vijfde vraag werd er gevraagd: \textit{Waarom kiest u voor agile?} om zo te kijken wat de beargumentering was en waarom er niet voor een andere methode werd gegaan.\newline  Als laatste werd er gevraagd : \textit{Hoelang duurt een gemiddelde sprint bij jullie?} om een beeld te hebben hoe men met tijd omgaat en hoelang een sprint duurt en hoe deze gelijkloopt of verschilt met bijvoorbeeld een sprint in de opleiding.



\section{Analyse van het interview}
Het eerste bedrijf dat positief reageerde op ons verzoek voor een interview was Flexsoft, een bedrijf gevestigd in Gent. Flexsoft is een software bedrijf dat zijn aandacht richt op twee grote software pakketen, Titanium en Themis. Deze software pakketen richten zich op tandartsen en advocaten. \newline \newline Ons contact persoon binnen Flexsoft was Glenn De Tollenaere. Glenn is mede oprichter van het bedrijf en werkt zelf al enkele jaren als full stack developer binnen het bedrijf. Na zijn werkuren gaf hij ons de kans een interview af te leggen over de werking van een business analist op agile projecten.\newline \newline Hij vertelde ons dat ze binnen Flexsoft al enkele jaren agile werken en dit de sfeer en productiviteit binnen het bedrijf een boost gegeven heeft. Binnen flexsoft werken ze in sprints van twee weken en starten ze elke dag met een korte standup meeting van maximum 15 minuten om te zorgen dat iedereen nog op schema zit. Op het einde van elke week doen ze een langere meeting om te bekijken waar ze zich moeten aanpassen om de sprint tot een goed einde te brengen. Deze meetings zorgen er niet enkel voor dat alles in de juiste richting verloopt maar voor de sfeer binnen het team zijn deze meetings ook erg belangrijk vertelde Glenn ons.\newline \newline Over de vragen over business analist moesten we wat extra uitleg geven, maar na enkele verduidelijkingen vertelde hij ons dat ze binnen Flexsoft geen aparte taak hebben voor business analist. Hij verwees ons wel door naar Pascal Desmet. Pascal is de team leader van Flexsoft, zijn taak bestaat eruit te zorgen dat de sprint in goede banen loopt en dat de product- en sprint backlog in orde zijn. Zelf is hij ook developer en helpt hij binnen het development team. Zijn belangrijkste taak en de reden waardoor Glenn ons doorverwees was dat hij in rechtstreeks contact stond met de klant en zijn taak daardoor vergeleken kon worden met een business analist.\newline \newline Na een leerrijk gesprek met Glenn kregen we ook de kans om een kort gesprek te voeren met Pascal die toevallig nog aanwezig was in het kantoor. Daar vertelde hij ons dat zijn taak inderdaad te vergeleken kan worden met business analist maar ze deze term binnen Flexsoft hebben samengevoegd met de team leader. Zo vertelde hij ook dat deze taak erg belangrijk is voor het goed verloop van een sprint en dat hij de werking van Flexsoft zicht niet meer kan inbeelden zonder met scrum te werken.\newline \newline Uit deze antwoorden kunnen we besluiten dat de taak van business analist makkelijk samengevoegd kan worden met andere taken binnen een agile project maar dat de functies van een business analist zeker belangrijk zijn binnen agile.\newline \newline Het tweede bedrijf zoeken ging wat minder vlot. Na een tijdje zoeken vonden we via linkedIn Rik Van Der Wardt een Agile coach en scrum master bij Agile Scrum Groep. Agile Scrum Group is een Nederlands bedrijf dat instaat voor het introduceren en implementeren van het agile werken in andere bedrijven. Wegens lange afstand werd de communicatie gedaan via e-mail, wat helemaal niet voor hinder zorgde. Agile Scrum Group was zeer behulpzaam en bezorgde veel nuttige informatie. Zo wisten ze te vertellen dat MIT(Massachusetts Institute of Technology) een recent onderzoek heeft gedaan naar de voordelen van agile werken en de impact ervan op een bedrijf.\newline \newline Uit de resultaten van dit onderzoek is gebleken dat bedrijven die het agile proces toepassen veel klantgerichter en innovatiever voor de dag komen, wat kan leiden tot een snellere groei van dit bedrijf van ongeveer 37\% en een significant hogere winst van ongeveer 30\%. \newline
\newline
Verder is ook vastgesteld dat agile werken niet alleen een grote impact heeft op het bedrijf zelf, maar ook op het personeel, die zich schijnbaar veel gelukkiger voelen, onder meer door de duidelijkheid en structuur dat agile werken met zich meebrengt.\newline
Specifieke antwoorden op de gestelde vragen werden niet gegeven. Maar met de informatie die werd meegegeven, kon er wel telkens een antwoord worden afgeleid. Zo beschreven ze de taak van een business-analist als een tussenpersoon tussen de klant en het projectteam. Een soort vertaler die de communicatie tussen de twee wat vlotter probeert te maken. Ze zorgen er dus met andere woorden voor dat hetgene de klant wil en het projectteam zal maken zo dicht mogelijk bij elkaar ligt. \newline
\newline
Wanneer er dan gekeken werd naar de vraag of business-analyse een skill of een rol was, kon er afgeleid worden dat het meer een skill was. Toch was dit anders dan verwacht. Zo wist men te zeggen dat een business-analist eigenlijk onbewust altijd agile werkt. Dit komt omdat van een business-analist verwacht wordt dat deze zeer flexibel is in zijn werk aangezien er rondom hem zoveel factoren zijn die kunnen veranderen.\newline
\newline
Voor het verloop van het agile proces werd een duidelijke structuur meegegeven. Het is zo dat een agile organisatie opgedeeld wordt in 5 sectoren die zeer goed moeten samenwerken. Zo heb je scrum teams, kanban teams, agile managers, agile coaches en agile Human Resources. Deze 5 dienen allemaal een agile mindset te volgen om het project te doen slagen. Er is dus geen specifieke rol voor een business-analist in het Scrum framework en zal zich moeten inleven in de rol van agile manager aangezien deze functie het dichtste aanleunt bij die van een business-analist \newline
\newline
De voor en nadelen van agile werken zijn hierboven al eens vermeld. Zo blijkt dat, uit een business oog bekeken, agile werken zeer lucratief is voor een bedrijf, maar ook zeer goed is voor de werknemers. Nadelen van agile werken werd niet gegeven. Dit komt waarschijnlijk door het feit dat Agile Scrum Group andere bedrijven wil beïnvloeden om agile te werken. Ze zijn als het ware verkopers van de agile method en hebben daarom waarschijnlijk enkel op de positieve factoren nadruk gelegd.\newline
\newline
Bij de vraag hoelang een sprint effectief duurt, werd er vermeld dat dit tussen de 1-3 weken ligt. Dit verschilt natuurlijk van project op project, maar is zo gekozen om de nodige focus te behouden in het project. Het is zo dat wanneer een sprint korter dan 1 week is, het bijna onmogelijk wordt om als werknemer een goed resultaat te kunnen waarmaken. Bij langer dan 3 weken kan het zijn dat de focus in het project wat kan verwateren. Daarom kozen ze ervoor de gouden middenweg te zoeken.

\section{Conclusie}
We concluderen dat we onze hypothese niet kunnen verwerpen, wat erop duidt dat men tegenwoordig eerder zonder business-analist aan projecten doet en dat de taken van de business-analist worden gecombineerd met een andere rol binnen het team.
Bij Flexsoft hebben ze bijvoorbeeld de taken die de business-analist doet, samengevoegd met team leader. \newline \newline
Daarnaast zegt de Agile Scrum Group dat deze tijd de business-analist geen rol meer is, maar eerder een set van skills. Zij zeggen dat de persoon die de klant met het projectteam verbindt, de business-analist is en recentelijk komt dat dus overeen met de taken van een team leader of project leader. \newline \newline
Binnen een agile omgeving is nu de agile methode Scrum een van de populairste. Hier wordt een project onderverdeeld in user stories die binnen een sprint van 1 tot 3 weken gevormd worden tot een feature van het project. \newline \newline
Alles kort samengevat komt het erop neer dat de skills van de business-analist in het huidige tijdperk niet meer een eigen rol krijgen in agile omgeving, maar dat deze steeds meer samen horen bij een andere rol die voor de communicatie zorgt tussen de klant en het team, zoals de team leader.


%------------------------------------------------------------------------------
% Referentielijst
%------------------------------------------------------------------------------
% TODO: de gerefereerde werken moeten in BibTeX-bestand ``bibliografie.bib''
% voorkomen. Gebruik JabRef om je bibliografie bij te houden en vergeet niet
% om compatibiliteit met Biber/BibLaTeX aan te zetten (File > Switch to
% BibLaTeX mode)

\phantomsection
\printbibliography

\end{document}
